\begin{lstlisting}[caption=開口測光用のコード, label=apphot, language=Python]
import glob
from astropy.io import fits
from astropy.time import Time
from photutils.aperture import CircularAperture, CircularAnnulus, aperture_photometry
from photutils.detection import DAOStarFinder
from astropy.stats import mad_std
import numpy as np
import matplotlib.pyplot as plt
from zoneinfo import ZoneInfo
from datetime import timezone
import pandas as pd
import matplotlib.dates as mdates

# 重心座標を検出するための関数を定義
def find_star_position(data, initial_pos, search_radius, fwhm=3.0, threshold=5.0):
    """
    初期位置近辺で星を検出し、正確な座標と検出された星の情報を返す関数
    """
    # 背景ノイズの推定
    bkg_sigma = mad_std(data)
    
    # 星の検出
    daofind = DAOStarFinder(fwhm=fwhm, threshold=threshold * bkg_sigma)
    sources = daofind(data - np.median(data))
    
    if sources is None or len(sources) == 0:
        print(f"No sources found around position {initial_pos}")
        return initial_pos, None  # 検出できなかった場合は初期位置とNoneを返す
    
    # 初期位置からの距離を計算
    distances = np.sqrt((sources['xcentroid'] - initial_pos[0])**2 + 
                        (sources['ycentroid'] - initial_pos[1])**2)
    
    # 検出範囲内の星をフィルタ
    within_radius = distances < search_radius
    if not np.any(within_radius):
        print(f"No sources within {search_radius}px of position {initial_pos}")
        return initial_pos, None  # 見つからない場合は初期位置とNoneを返す
    
    # 検出された星の中で最も明るいものを選択
    brightest = sources[within_radius][sources['flux'][within_radius].argmax()]
    
    return (brightest['xcentroid'], brightest['ycentroid']), brightest
    
# 検索範囲 (pxl) 
search_radius = 15

# 結果を保存するリスト
times = []
rel_mags = []
errors = []
target_mags = []

# ターゲット星, 参照星の座標の初期値を入力
target_initial = (376.00, 295.00)
comp_initials = [
    (477.00, 324.00), 
    (474.00, 250.00), 
    (310.00, 190.00), 
    (227.00, 371.00)
]

# 参照星の数を取得
num_comps = len(comp_initials)

# 各参照星のフラックスを保存するリストのリストを作成
comp_fluxes_list = [[] for _ in range(num_comps)]


# FITSファイルのリストを取得(日時順にソート)
fits_files = sorted(glob.glob('./data/10toi1516a010*.fit'))

# ターゲット星と参照星の現在の位置
target_position = target_initial
comp_positions = comp_initials.copy()

# 各画像での処理
for idx, file in enumerate(fits_files):
    print(f"Processing {file} ({idx+1}/{len(fits_files)})")
    try:
        with fits.open(file) as hdu:
            data = hdu[0].data
            header = hdu[0].header
    except Exception as e:
        print(f"Error opening {file}: {e}")
        for flux_list in comp_fluxes_list:
            flux_list.append(np.nan)
        target_mags.append(np.nan)
        continue
    
    # 観測時刻を取得
    if 'DATE-OBS' in header:
        time_utc = Time(header['DATE-OBS']).to_datetime(timezone=timezone.utc)
        time_jst = time_utc.astimezone(ZoneInfo('Asia/Tokyo'))
    else:
        print(f"DATE-OBS not found in {file}. Skipping.")
        for flux_list in comp_fluxes_list:
            flux_list.append(np.nan)
        target_mags.append(np.nan)
        continue
    times.append(time_jst)
    
    # ターゲット星の位置と明るい星の情報を取得
    target_position, brightest_star = find_star_position(data, target_position, search_radius=search_radius)
    
    # FWHMの取得
    if brightest_star is not None and 'fwhm' in brightest_star:
        measured_fwhm = brightest_star['fwhm']
        aperture_radius = measured_fwhm * 1.5  # FWHMの1.5倍をアパーチャ半径とする
    else:
        aperture_radius = 6  # デフォルト値
    
    # 参照星の位置を検索
    for i, comp_pos in enumerate(comp_positions):
        comp_positions[i], _ = find_star_position(data, comp_pos, search_radius=search_radius)
    
    # 測光アパーチャーを定義(動的に設定した半径を使用)
    positions = [target_position] + comp_positions
    apertures = CircularAperture(positions, r=aperture_radius)
    
    # アパーチャで測光を実行
    phot_table_apertures = aperture_photometry(data, apertures)
    
    # 背景補正用アノラスを定義(各アパーチャに対して1つのアノラスを作成)
    annuli = CircularAnnulus(positions, r_in=aperture_radius + 1, r_out=aperture_radius + 4)
    
    # アノラスで測光を実行
    phot_table_annuli = aperture_photometry(data, annuli)
    
    # 各アパーチャに対応するアノラスのフラックスを取得
    # annuli.areaは全アノラスの面積のリストを返す
    bkg_fluxes = phot_table_annuli['aperture_sum'] / annuli.area * apertures.area
    
    # アパーチャ内の背景を差し引いたフラックスを計算
    target_flux = phot_table_apertures['aperture_sum'][0] - bkg_fluxes[0]
    comp_fluxes = phot_table_apertures['aperture_sum'][1:] - bkg_fluxes[1:]
    
    # フラックスが負になる場合は np.nan に設定
    target_flux = target_flux if target_flux > 0 else np.nan
    comp_fluxes = np.where(comp_fluxes > 0, comp_fluxes, np.nan)
    
    # 相対等級を計算
    if not np.isnan(target_flux) and not np.isnan(comp_fluxes).all():
        comp_flux = np.nanmean(comp_fluxes)
        if comp_flux > 0:
            rel_mag = -2.5 * np.log10(target_flux / comp_flux)
        else:
            rel_mag = np.nan
    else:
        rel_mag = np.nan
    rel_mags.append(rel_mag)

    # 機械等級を計算
    if not np.isnan(target_flux) and target_flux > 0:
        target_mag = -2.5 * np.log10(target_flux)
    else:
        target_mag = np.nan
    target_mags.append(target_mag)
    
    # 誤差の推定
    if not np.isnan(comp_flux):
        error = np.nanstd(comp_fluxes / comp_flux)
    else:
        error = np.nan
    errors.append(error)
    
    # 各参照星のフラックスをリストに追加
    for i, flux in enumerate(comp_fluxes):
        comp_fluxes_list[i].append(flux)
        print(f'Comparison Star {i+1}: Flux = {flux}')
\end{lstlisting}