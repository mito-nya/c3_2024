\begin{lstlisting}[caption=トランジットモデル計算およびプロット用のコード, label=model, language=Python]
    dt = datetime.datetime(2024, 11, 11, 21, 22, 00)

    # dt をユリウス日に変換
    t = Time(dt, scale='utc')
    t0_jd = jd_value
    
    # Transit parameter 設定
    params = batman.TransitParams()
    params.t0 = t0_jd        # トランジット中心を JD で設定
    params.per = 2.056014
    params.rp = 0.1224
    params.a = 6.220
    params.inc = 90.0
    params.ecc = 0.0
    params.w = 90.0
    params.u = []           
    params.limb_dark = "uniform"
    
    time_jd = np.linspace(params.t0 - 0.07, params.t0 + 0.07, 5000)
    
    # トランジットモデルの計算
    m = batman.TransitModel(params, time_jd)
    flux_model = m.light_curve(params)
    
    # datetimeに変換しておく
    time_datetime = Time(time_jd, format='jd').to_datetime()
    
    # ここからはデータのプロット
    times_jst_naive = [dt.replace(tzinfo=None) if dt is not None and not isinstance(dt, float) else dt for dt in times]
    
    # データをDataFrameに変換
    df = pd.DataFrame({
        'Time': times_jst_naive,
        'Flux_Norm': norm_fluxes
    })
    
    # 'Time'をDatetimeIndexに設定
    df.set_index('Time', inplace=True)
    time_array = df.index.to_numpy()
    
    # 光度曲線のプロット
    fig, ax = plt.subplots(figsize=(10, 6))
    plt.plot(time_array, df['Flux_Norm'], 'o', markersize=3, alpha=0.3, color='black', label="data")
    plt.plot(time_datetime, flux_model, color='black', lw=3, label="model")
    ax.xaxis.set_major_formatter(mdates.DateFormatter('%H:%M'))
    # plt.gca().invert_yaxis()
    plt.xlabel('Time')
    plt.ylabel('Flux [Rel.]')
    plt.title('TOI-1516 Light Curve')
    plt.grid(True)
    plt.show()
    fig.savefig('light_curve_fit.png')
\end{lstlisting}